\documentclass[10pt,twocolumn,oneside]{article}
\setlength{\columnsep}{18pt}                    %兩欄模式的間距
\setlength{\columnseprule}{0pt}                 %兩欄模式間格線粗細

\usepackage{amsthm}                             %定義,例題
\usepackage{amssymb}
\usepackage{fontspec}                           %設定字體
\usepackage{color}
\usepackage[x11names]{xcolor}
\usepackage{listings}                           %顯示code用的
\usepackage{fancyhdr}                           %設定頁首頁尾
\usepackage{graphicx}                           %Graphic
\usepackage{enumerate}
\usepackage{titlesec}
\usepackage{amsmath}
\usepackage[CheckSingle, CJKmath]{xeCJK}
\usepackage{CJKulem}

\usepackage{amsmath, courier, listings, fancyhdr, graphicx}
\topmargin=0pt
\headsep=5pt
\textheight=740pt
\footskip=0pt
\voffset=-50pt
\textwidth=545pt
\marginparsep=0pt
\marginparwidth=0pt
\marginparpush=0pt
\oddsidemargin=0pt
\evensidemargin=0pt
\hoffset=-42pt

%\renewcommand\listfigurename{圖目錄}
%\renewcommand\listtablename{表目錄}

%%%%%%%%%%%%%%%%%%%%%%%%%%%%%

\setmainfont[
    AutoFakeSlant,
    BoldItalicFeatures={FakeSlant},
    UprightFont={* Medium},
    BoldFont={* Bold}
]{Inconsolata}
%\setmonofont{Ubuntu Mono}
\setmonofont[
    AutoFakeSlant,
    BoldItalicFeatures={FakeSlant},
    UprightFont={* Medium},
    BoldFont={* Bold}
]{Inconsolata}
\setCJKmainfont{Noto Sans CJK TC}
\XeTeXlinebreaklocale "zh"                      %中文自動換行
\XeTeXlinebreakskip = 0pt plus 1pt              %設定段落之間的距離
\setcounter{secnumdepth}{3}                     %目錄顯示第三層

%%%%%%%%%%%%%%%%%%%%%%%%%%%%%
\makeatletter
\lst@CCPutMacro\lst@ProcessOther {"2D}{\lst@ttfamily{-{}}{-{}}}
\@empty\z@\@empty
\makeatother
\lstset{                                        % Code顯示
    language=C++,                               % the language of the code
    basicstyle=\footnotesize\ttfamily,          % the size of the fonts that are used for the code
    numbers=left,                               % where to put the line-numbers
    numberstyle=\scriptsize,                    % the size of the fonts that are used for the line-numbers
    stepnumber=1,                               % the step between two line-numbers. If it's 1, each line  will be numbered
    numbersep=5pt,                              % how far the line-numbers are from the code
    backgroundcolor=\color{white},              % choose the background color. You must add \usepackage{color}
    showspaces=false,                           % show spaces adding particular underscores
    showstringspaces=false,                     % underline spaces within strings
    showtabs=false,                             % show tabs within strings adding particular underscores
    frame=false,                                % adds a frame around the code
    tabsize=2,                                  % sets default tabsize to 2 spaces
    captionpos=b,                               % sets the caption-position to bottom
    breaklines=true,                            % sets automatic line breaking
    breakatwhitespace=true,                     % sets if automatic breaks should only happen at whitespace
    escapeinside={\%*}{*)},                     % if you want to add a comment within your code
    morekeywords={*},                           % if you want to add more keywords to the set
    keywordstyle=\bfseries\color{Blue1},
    commentstyle=\itshape\color{Red1},
    stringstyle=\itshape\color{Green4},
}


\begin{document}
\pagestyle{fancy}
\fancyfoot{}
%\fancyfoot[R]{\includegraphics[width=20pt]{ironwood.jpg}}
\fancyhead[C]{team053}
\fancyhead[L]{FJCU_CSIE_O4O_CodeBook}
\fancyhead[R]{\thepage}
\renewcommand{\headrulewidth}{0.4pt}
\renewcommand{\contentsname}{Contents}

\scriptsize
\tableofcontents
\section{C語言}
    \subsection{basic}
        \lstinputlisting{Contents/C語言/basic.cpp}
    \subsection{字元}
        #include<ctype.h>

讀取字元,可用 getchar()
x = getchar();
列印字元,可用 putchar()
putchar(x);
putchar('\n');

int isalnum(int c) 檢查傳遞的字符是否是字母數字。
int isalpha(int c) 是否傳遞的字符是字母。
int iscntrl(int c) 是否傳遞的字符是控制字符。
int isdigit(int c) 是否傳遞的字符是十進制數字。
int islower(int c) 檢查傳遞的字符是否是小寫字母。
int isprint(int c) 檢查傳遞的字符是否是可打印的。
int ispunct(int c) 檢查傳遞的字符是否是標點符號。
int isspace(int c) 檢查傳遞的字符是否是空白。
int isupper(int c) 檢查傳遞的字符是否是大寫字母。
int isxdigit(int c) 檢查傳遞的字符是否是十六進製數字。
    \subsection{字串}
        #include<string.h>

strcpy	將字串 s2 拷貝到 s1	char *strcpy(char *s1, const char *s2);
strncpy	將字串 s2 最多 n 個字元拷貝到 s1	char *strncpy(char *s1, const char *s2, size_t n);
strcat	將字串 s2 接到 s1 的尾端	char *strcat(char *s1, const char *s2);
strncat	將字串 s2 最多 n 個字元接到 s1 的尾端	char *strncat(char *s1, const char *s2, size_t);
strcmp	比較 s1 與 s2 兩個字串是否相等	int strcmp(const char *s1, const char *s2);
strncmp	比較 s1 與 s2 兩個字串前 n 個字元是否相等	int strncmp(const char *s1, const char *s2, size_t n);
strcspn	計算經過幾個字元會在字串 s1 中遇到屬於 s2 中的字元	size_t strcspn(const char *s1, const char *s2);
strspn	計算經過幾個字元會在字串 s1 中遇到不屬於 s2 中的字元	size_t strspn(const char *s1, const char *s2);
strpbrk	回傳在字串 s2 中的任何字元在 s1 第一次出現位置的指標	char *strpbrk(const char *s1, const char *s2);
strchr	回傳在字串 s 中,字元 c 第一次出現位置的指標	char *strchr(const char *s, int c);
strrchr	回傳在字串 s 中,字元 c 最後一次出現位置的指標	char *strrchr(const char *s, int c);
strstr	回傳在字串 s2 在 s1 第一次出現位置的指標	char *strstr(const char *s1, const char *s2);
strtok	以字串 s2 的內容切割 s1	char *strtok(char *s1, const char *s2);
strlen	計算字串的長度	size_t strlen(const char *s);

memcpy	從 s2 所指向的資料複製 n 個字元到 s1	void *memcpy(void *s1, const void *s2, size_t n);
memmove	從 s2 所指向的資料複製 n 個字元到 s1	void *memmove(void *s1, const void *s2, size_t n);
memcmp	比較 s1 與 s2 前 n 個字元的資料	int memcmp(const void *s1, const void *s2, size_t n);
memchr	找出字元 c 在 s 前 n 個字元第一次出現的位置	void *memchr(const void *s, int c, size_t n);
memset	將 s 中前 n 個字元全部設定為 c	void *memset(void *s, int c, size_t n);

    \subsection{型別範圍}
        int                 -2,147,483,648 ~ 2,147,483,647
unsigned int	    0 ~ 4,294,967,295
char                -128 ~ 127
unsigned char       0 ~ 255
long long           -9,223,372,036,854,775,808 ~ 9,223,372,036,854,775,807
unsigned long long  0 ~ 18,446,744,073,709,551,615
float               3.4E +/- 38 (7 位數)
double              1.7E +/- 308 (15 位數)
    \subsection{格式控制字串}
        printf()
%ld 長整數
%lld long long整數
%Lf long double
%o 無號八進位整數
%u 無號十進位整數
%x 無號十六進位整數
scanf()
%lf 被精度浮點數
%Lf long double
    \subsection{math}
        #include<math.h>

double acos(double x) 返回x的反餘弦弧度。
double asin(double x) 返回x的正弦弧線弧度。
double atan(double x) 返回x的反正切值,以弧度為單位。
double atan2(doubly y, double x) 返回y / x的以弧度為單位的反正切值,根據這兩個值,以確定正確的象限上的標誌。
double cos(double x) 返回的弧度角x的餘弦值。
double cosh(double x) 返回x的雙曲餘弦。
double sin(double x) 返回一個弧度角x的正弦。
double sinh(double x) 返回x的雙曲正弦。
double tanh(double x) 返回x的雙曲正切。
double exp(double x) 返回e值的第x次冪。
double frexp(double x, int *exponent)
The returned value is the mantissa and the integer yiibaied to by exponent is the exponent. The resultant value is x = mantissa * 2 ^ exponent.
double ldexp(double x, int exponent)
Returns x multiplied by 2 raised to the power of exponent.
double log(double x) 返回自然對數的x(基準-E對數)。
double log10(double x) 返回x的常用對數(以10為底)。
double modf(double x, double *integer) 返回的值是小數成分(小數點後的部分),並設置整數的整數部分。
double pow(double x, double y) 返回x的y次方。
double sqrt(double x) 返回x的平方根。
double ceil(double x) 返回大於或等於x的最小整數值。
double fabs(double x) 返回x的絕對值
double floor(double x) 返回的最大整數值小於或等於x。
double fmod(double x, double y) 返回的x除以y的餘數。

#include<stdlib.h>

int abs(int x) 返回x的絕對值。
div_t div(int numer, int denom) 數(分子numer)除以分母(分母denom)。
long int labs(long int x) 返回x的絕對值。
ldiv_t ldiv(long int numer, long int denom) 數(分母denom)除以分母(分子numer)。
int rand(void) 返回一個取值範圍為0到RAND_MAX之間的偽隨機數。
void srand(unsigned int seed) 這個函數使用rand函數隨機數生成器的種子。

\section{Section2}
    \subsection{thm}
        \input{Contents/section2/thm.tex}

\end{document}
