#include<math.h>

double acos(double x) 返回x的反餘弦弧度。
double asin(double x) 返回x的正弦弧線弧度。
double atan(double x) 返回x的反正切值,以弧度為單位。
double atan2(doubly y, double x) 返回y / x的以弧度為單位的反正切值,根據這兩個值,以確定正確的象限上的標誌。
double cos(double x) 返回的弧度角x的餘弦值。
double cosh(double x) 返回x的雙曲餘弦。
double sin(double x) 返回一個弧度角x的正弦。
double sinh(double x) 返回x的雙曲正弦。
double tanh(double x) 返回x的雙曲正切。
double exp(double x) 返回e值的第x次冪。
double frexp(double x, int *exponent)
The returned value is the mantissa and the integer yiibaied to by exponent is the exponent. The resultant value is x = mantissa * 2 ^ exponent.
double ldexp(double x, int exponent)
Returns x multiplied by 2 raised to the power of exponent.
double log(double x) 返回自然對數的x(基準-E對數)。
double log10(double x) 返回x的常用對數(以10為底)。
double modf(double x, double *integer) 返回的值是小數成分(小數點後的部分),並設置整數的整數部分。
double pow(double x, double y) 返回x的y次方。
double sqrt(double x) 返回x的平方根。
double ceil(double x) 返回大於或等於x的最小整數值。
double fabs(double x) 返回x的絕對值
double floor(double x) 返回的最大整數值小於或等於x。
double fmod(double x, double y) 返回的x除以y的餘數。

#include<stdlib.h>

int abs(int x) 返回x的絕對值。
div_t div(int numer, int denom) 數(分子numer)除以分母(分母denom)。
long int labs(long int x) 返回x的絕對值。
ldiv_t ldiv(long int numer, long int denom) 數(分母denom)除以分母(分子numer)。
int rand(void) 返回一個取值範圍為0到RAND_MAX之間的偽隨機數。
void srand(unsigned int seed) 這個函數使用rand函數隨機數生成器的種子。
怎麼會這樣拉